\chapter{\abstractname}
As obesity becomes more and more of a problem in developed countries, food logging is frequently used to help overweight people to balance their energy intake. Unfortunately, food logging is a tedious and inaccurate process. Computer vision and machine learning can help the user with this process. By taking an image of the meal, algorithms are able to make automated food intake assessments by detecting food items and their size. The goal of this thesis is the evaluation and implementation of a proof of concept application that can be used to facilitate and extend future food logging applications.

For the task of food classification seven approaches were evaluated including feature classifiers like SIFT and SURF and convolutional neural networks. To enable the classification, segmentation, training and evaluation of classifiers, an extensive application was implemented using Python. The application supports data preprocessing and is designed so that it can be extended with additional image recognition concepts.

With the aforementioned algorithms it was shown that algorithms are able to achieve a classification accuracy of 75\% on 50 different food item classes if the algorithm suggests five possible candidates.


\chapter{Zusammenfassung}
Fettleibigkeit wird zunehmend zu einem Problem in Industrieländern. Durch das Führen eines Ernährungstagebuchs kann übergewichtigen Personen dabei geholfen werden, ihren Energiehaushalt auszugleichen. Ernährungstagebücher sind jedoch oft ungenau und das detaillierte Führen eines solchen Tagebuchs ist aufwendig. Durch Computer Vision und Machine Learning Konzepte kann dieser Prozess erleichtert werden. Nutzer können ein Bild ihrer Mahlzeit aufnehmen, welches durch Algorithmen analysiert wird. Diese können dann automatisiert den Energiewert berechnen und in das Tagebuch eintragen. Das Ziel dieser Arbeit ist die Evaluierung und Implementierung eines Machbarkeitsnachweises, welcher für zukünftige Ernährungstagebücher benutzt und erweitert werden kann.

Für die Klassifizierung der Nahrungsmittel wurden sieben Ansätze getestet. Unter anderem wurden SIFT, SURF und Convolutional Neural Networks geprüft. Dafür entstand ein umfassendes Programm, welches Klassifizierung, Segmentierung und die Evaluierung von verschiedenen Bilderkennungsalgorithmen ermöglicht. Das Programm unterstützt zudem diverse Algorithmen zur Datenvorverarbeitung und ist so konzipiert, dass es einfach mit weiteren Konzepten zur Bilderkennung erweitert werden kann. 

Mit den genannten Algorithmen konnte gezeigt werden, dass sich eine Genauigkeit von 75\% erreichen lässt, wenn der Algorithmus die Möglichkeit erhält, eine Liste mit 5 Vorschlägen als Antwort zu geben.





